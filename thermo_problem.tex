\chapter{Thermal field} \label{ch:thermal_problem}

For the development of the thermomechanical models, the temperature field needs to be considered.
This section provides an overview of the governing equations required to describe a temperature field, including their spatial discretization using the finite element method (FEM).
The procedure to establish a fully discrete system of equations for the themal field is comparable to the one for the mechanical field in Chapter~\ref{ch:mechanical_problem}.
Furthermore, the basics of nonlinear continuum thermodynamics have already been featured in Chapter~\ref{ch:continuum_mechanics}.
Consequently, a detailed derivation is skipped in this chapter.

% In a first step, the balance equations for the thermal field will be established. Then, in a second step the thermal initial boundary value problem (IBVP) will be presented followed by the numerical solution technique. Latter requires a weak form of the thermal balance equation which will be fully discretised using the FEM for space discretisation and the finite difference method for time discretisation. To finish, the residual and the tangential system matrix will be introduced to enable the application of a Newton-Raphson method.

\section{Governing equations}

Based on the general model presented in Section~\ref{sec:fundamental_conservation_princ}, the balance equations for the temperature field are obtained as a special case by neglecting all mechanical terms.
Hence, the energy balance equation (Equation~\eqref{eq:first_principle_thermo}), now in material description, reduces to
\begin{equation} \label{eq:strong_energy_eq}
\rho_0\ e=-\operatorname{div}_0 \bm q_0+\rho_0 r \quad \text { in } \Omega_0,
\end{equation}
where all mechanical terms are neglected.
The target application of the present work are coupled generally nonlinear thermomechanical interaction problems, where the initial and the current domains are not equal, i.e. \(\Omega_{0} \neq \Omega\).
Thus, for the sake of simplicity and in view of the later coupled problem, all following relations are expressed in material quantities.
A purely thermal analysis is independent of the deformation, so that reference and current configuration are identical and the domain remains constant, i.e. \(\Omega_{0} \equiv \Omega\).

\section{Thermal constitutive initial value problem}

From Section~\ref{sec:thermomechanical_constitutive}, discarding all variables related to the mechanical problem, the general thermal constitutive initial value problem is
\begin{problem}[General thermal constitutive intial value problem.]
GGiven the initial value of the internal variables \(\bm \alpha(t_0)\) and the history of the temperature distribution and its gradient
\begin{gather}
\theta(t),\quad t\in[t_0, t_\text{end}],\\
\bm g_0(t), \quad t\in[t_0, t_\text{end}],
\end{gather}
find the function for $\bm q_0(t)$, and \(\bm \alpha(t)\) such that the constitutive equations
\begin{gather}
    s = -\frac{\partial \psi}{\partial \theta},\label{eq:entropy_constitutive_relation}\\
    \psi = \psi(\theta, \bm \alpha),\\
    \dot{\bm \alpha} = f(\theta, \bm g_0, \bm \alpha),\\
    \frac{1}{\theta}\bm q_0 = g(\theta, \bm g_0, \bm \alpha).
\end{gather}
are satisfied for every $t\in [t_0, t_\text{end}]$.
\end{problem}
No distinction between spatial and material configurations applies as \(\Omega = \Omega_0\).
Next, a standard set of assumptions are introduced.

\paragraph{Helmholtz free energy}
As a first step, the specific heat \(C_{V}\) is established and defined according to the thermodynamical principles to be the amount of heat required to change a unit mass of a substance by one degree in temperature, i.e.
\begin{equation}
C_{\mathrm{V}}=\frac{\partial e}{\partial \theta}.
\end{equation}
The index \((\cdot)_\mathrm{V}\) denotes that \(C_\mathrm{V}\) is measured at constant volume \footnote{According to \cite{cengel_thermodynamics_2011} the specific heat at constant volume and constant pressure for solids can be suitable assumed to be equal.}.
Its dimensions are energy over temperature, i.e., \(\mathrm{[E/\Theta]}\), and using the International System of Units (SI), \(C_{V}\) is expressed in joule per kelvin.
Using Equation~\ref{eq:def_helmholtz_free_energy}, the specific heat at constant volume can be written as
\begin{equation} \label{eq:def_cv_partial}
C_{\mathrm{V}}=-\frac{\partial^{2} \psi}{\partial \theta^{2}} \theta=\frac{\partial s}{\partial \theta} \theta.
\end{equation}
% In general, the heat capacity depends on the deformation and on the temperature.
% A substance whose specific volume (or density) is constant is called an incompressible substance.
% This incompressibilty or constant-volume assumption should be taken to imply that the energy associated with the volume change is negligible compared with other forms of energy.
% For the application to elastomers, see for instance Netz [96], the heat capacity \(C_{V}\) can be assumed to depend only on the temperature, and the partial derivatives in Equation~\eqref{eq:def_cv_partial} turn into exact derivatives, yielding
% \begin{equation}
%   C_\mathrm{V} = \frac{\ud s}{\ud \theta}\theta.
% \end{equation}
Furthermore, for the application to metals, a constant specific heat capacity (i.e. \(C_{\mathrm{V}}=\) const.) is a valid assumption, utilised e.g. in \cite{adam_thermomechanical_2005}, \cite{ghadiani2005multiphasic}, \cite{ibrahimbegovic_covariant_2002}, and \cite{simo_associative_1992}.
% Accordingly, the heat capacity is also assumed to be constant (i.e. \(C_{\mathrm{V}}= \mathrm{const}\).), since focus in this work is on the application to metals.
% Thus, the entropy can be written as
% \begin{highlight}
% \begin{equation}
% s(\theta) = C_\mathrm{V}\ln\left(\frac{\theta}{\theta_0}\right),
% \end{equation}
% \end{highlight}
% after integration, where \(\theta_{0}\) and \(C_{\mathrm{V}}\) denote the constant initial temperature and the constant specific heat, respectively.
%
% Given the constituive relation for the entropy (Equation~\eqref{eq:entropy_constitutive_relation}), the Helmholtz free energy per unit reference volume is found to be
% \begin{highlight}
% \begin{equation}
% \psi(\theta)=- C_{\mathrm{V}}\left[\left(\theta-\theta_{0}\right)-\theta \ln \left(\frac{\theta}{\theta_{0}}\right)\right],
% \end{equation}
% \end{highlight}
% Subsequently, the time derivative of the entropy is
% \begin{equation}
% \dot{s}(\theta)=\frac{\partial s}{\partial \theta} \dot{\theta}=-\frac{\partial^{2} \psi}{\partial \theta^{2}} \dot{\theta}=C_{\mathrm{V}} \frac{1}{\theta} \dot{\theta}.
% \end{equation}

\paragraph{Law for the heat flux}

Since deformation is not taken into account in purely thermal analyses, the material's heat flux and the heat flux in space coincide, or \(\bm q_0 \equiv \bm q\).
This relationship also holds true for both the material gradient and the spatial gradient, therefore \(\nabla_0 \theta \equiv \nabla \theta\).
A constitutive law for the heat flow must be determined that links the heat flux \(\bm q_0\) with its dual variable \(\bm g_0\) and the temperature \(\theta\) in order to meet the dissipation inequality due to conduction (Equation~\eqref{eq:dissipation_conduction}).
In light of this, the so-called Fourier's law, which is linear and isotropic and is stated as
\begin{highlight}
\begin{equation} \label{eq:def_fourier_law}
  \bm q_0=-k \bm g_0.
\end{equation}
\end{highlight}
The thermal conductivity \(k\) is considered to be constant and positive in this context, that is \(k \geq 0\).
Heat is therefore transferred in the direction of decreasing temperatures.
In addition to Fourier's law, there are several constitutive laws for the heat flux that may be found in the literature.
One such rule is Duhamel's law of heat conduction (see e.g. \cite{holzapfel_nonlinear_2000}), which swaps the scalar constant conductivity for a positive semi-definite second-order tensor.
The conductivity tensor simplifies to \(\bm k=k \bm I\) if Duhamel's law is constrained to thermally isotropic behavior (i.e. no preferred direction).
Fourier's law is recovered as a particular case of Duhamel's law if a constant heat conductivity is assumed.
But in the current work, Fourier's law is the only one taken into account since it is a simple law that yields physical effects.

\paragraph{"Standard" thermal constitutive description}
No extra internal variables \(\bm \alpha\) are considered in the present description of the thermal problem.
Thus, the thermal constitutive initial value problem given the standard assumptions laid out above accepts a closed form solution.

\begin{problem}["Standard" thermal constitutive description]
GGiven the history of the temperature distribution and its gradient
\begin{gather}
  \theta(t), \quad t\in [t_0, t_\text{end}],\\
  \bm g_0(t), \quad t \in [t_0, t_\text{end}],
\end{gather}
compute the $\bm q_0(t)$ and \(s(t)\) at every $t\in [t_0, t_\text{end}]$ using the constitutive equations
\begin{gather}
    s = - \frac{\partial \psi}{\partial \theta},\\
    \bm q_0 = -k\bm g_0.
\end{gather}
\end{problem}

\section{Weak energy balance equation}

The weak form of the energy balance equation must be used in order to solve the thermal issue using the FEM.
The energy balance equation may be found by taking the governing equation (Equation~\eqref{eq:strong_energy_eq}) in its strong form, using a variational strategy, multiplying it by the virtual temperatures \(\xi\), and then integrating by parts.
\begin{problem}[Weak energy balance equation]
TThere is energy balance in the body if and only if the temperature distribution satisfies
    \begin{equation}
        \int_{\Omega_0}   \left[\left(\dot e - \rho_0 r\right) \xi - \bm q_0\cdot \nabla_0 \xi\right]\ud v - \int_{\partial\Omega_0} h_0 \xi\ud a = 0,\quad \forall \xi \in \mathscr{V}_{\theta,0},
    \end{equation}
 where $\mathscr{V}_{\theta,0}$ is the space of virtual temperature distributions on the body, defined by the space of sufficiently regular arbitrary temperature distributions.
 \end{problem}

\section{The thermal initial boundary value problem}

Following the same approach as in Section~\ref{sec:mechanical_constitutive_problem}, it is now possible to introduce the the thermal initial boundary value problem.
Assume that the internal variables governing the body \(\mathcal B\) are known at the initial time \(t_0\).
In addition, assume that the heat generated in the interior of the body is prescribed, \(r(\bm X, t)\), \(t\in[t_0, t_\text{end}]\), as well as,
\begin{itemize}
  \item \textbf{Natural (or Neumann) boundary condition.} The boundary portion \(\partial \Omega_\text{heat,0}\) of \(\mathcal B\) is subject to a prescribed history of heat flux, \(h_\text{presc,0}(\bm X, t) = \bm q_\text{presc,0}(\bm X, t)\cdot \bm m(\bm X)\), \(\bm X \in \partial \Omega_\text{heat,0}\), \(t\in [t_0,t_\text{end}]\).
  \item \textbf{Essential (or Dirichlet) boundary condition.} The boundary portion \(\partial \Omega_\text{temperature,0}\) of \(\mathcal B\) is subject to a prescribed temperature history, \(\theta_\text{presc}(\bm X, t)\), \(\bm X \in \partial \Omega_\text{temperature,0}\), \(t\in [t_0,t_\text{end}]\).
\end{itemize}

The body's allowable temperature distributions, as before, are all the sufficiently regular temperature fields that meet the necessary boundary condition, i.e.,
\begin{equation}
  \mathscr K_\theta = \{\theta:\Omega_0 \times \mathbb R \to \mathbb R\,|\,\theta(\bm X, t) = \theta_\text{presc}(\bm X, t),\quad \bm X\in \partial \Omega_\text{temperature,0},\quad t\in[t_0, t_\text{end}]\}.
\end{equation}

Heat convection and radiation are the natural boundary conditions for the thermal problem studied in this work.
The prescribed heat at the boundary in the first case is given as
\begin{equation} \label{eq:heat_convection}
  h_\text{presc,0}(\bm X, t) = h_c(T(\bm X, t) - T_\infty),\quad \mathbf X \in \partial \Omega_{\text{convection,0}}\subseteq \partial\Omega_{\text{heat,0}}
\end{equation}
with \(h_c\) denoting the coefficient of heat transfer by covection and \(T_\infty\) is the temperature of the environment.
The appropriate statement for heat transmission by radiation is
\begin{equation}
  h_\text{presc, 0} (\bm X, t) =  \epsilon \sigma T^4,\quad \mathbf X \in \partial \Omega_{\text{radiation,0}}\subseteq \partial\Omega_{\text{heat,0}}
\end{equation}
where \(\sigma=\SI{5.670373(21)e-8}{\watt\meter^{-2}\kelvin^{-4}}\) is Boltzman's constant and \(\epsilon\) is the emissivity factor.


Using the definition of Helmholtz's energy (Equation~\ref{eq:def_helmholtz_free_energy}), the balance of energy (Equation~\eqref{eq:strong_energy_eq}) can be written as
\begin{equation}
  \rho_0\dot \psi + \rho_0 \dot \theta s + \rho_0 \theta \dot s = \rho_0 r - \operatorname{div}\bm q_0.
\end{equation}
Applying the chain rule to \(\dot s\) yields
\begin{equation}
  \rho_0 \dot \psi + \rho_0 \dot \theta s + \rho_0 \theta \frac{\partial s}{\partial \theta}\dot \theta  = \rho_0 r - \operatorname{div}\bm q_0.
\end{equation}
Applying Equation~\eqref{eq:def_cv_partial} regarding the specific heat at constant volume reveals
\begin{equation}
  \rho_0 (\dot \psi +  \dot \theta s) + \rho_0 C_V\dot \theta = \rho_0 r - \operatorname{div}\bm q_0.
\end{equation}
According to Equation~\eqref{eq:def_int_dissipation}, the term \(\dot \psi + \dot \theta s\) is precisely the internal dissipation \(\pazocal D_\text{int}\), which for a strictly thermal process is zero.
Accounting also for the constitutive law chosen for the heat flux, i.e., Fourier's law (Equation~\eqref{eq:def_fourier_law}), one finally obtains
\begin{equation}
  \rho_0 C_V \dot \theta = - \rho_0 r + k\operatorname{div} \bm g_0.
\end{equation}

The the weak form of the thermal constitutive initial boundary value problem can thus be stated as follows
 \begin{problem}[Thermal initial BVP.]
     FFind an admissible temperature distribution, $\theta \in \mathscr{K}_\theta$, such that for every $t\in [t_0,t_\text{end}]$, the body $\mathscr{B}$ satisfies energy conservation
         \begin{equation}
         \int_{\Omega_0}   \left[\left(\rho_0 C_\mathrm{V} \dot \theta - \rho_0 r\right) \xi +k\bm g_0\cdot \nabla_0 \xi\right]\ud v - \int_{\partial\Omega_0} h_0 \xi\ud a = 0,\quad \forall \xi \in \mathscr{V}_{\theta,0},
     \end{equation}
     where the space of virtual temperature distributions at time $t$ is defined by
     \begin{equation}
         \mathscr{V}_{\theta,0} \equiv \left\{\xi:\Omega_0\to \mathbb R\;|\;\xi = 0\quad \text{in}\quad \partial\Omega_\text{temperature,0}\right\}
     \end{equation}
     and at each point of $\mathscr{B}$.
 \end{problem}

\section{Finite Element Method} \label{sec:fem_therm}

Following a procedure entirely similar to the one described in Section~\ref{sec:fem_mech}, the global shape functions can be conveniently assembled in the so-called global interpolation vector as
\begin{equation}
\mathbf{N}^{g}(\boldsymbol{X}) \equiv\left[N_1^g(\bm X), N_2^g(\bm X), \dots, N^g_{n_\text{points}}(\bm X)\right].
\end{equation}

The vector containg the nodal values of the temperature is denoted by \(\bm \uptheta\) and defined as
\begin{equation}
 \bm \uptheta (t)= \left[\theta^{1}(t), \dots, \theta^{n_{\text {points}}}(t)\right]^{T},
\end{equation}
such that the value of the temperature inside the descretized domain \(^h\Omega_0\) can be found from
\begin{highlight}
\begin{equation}
{ }^{h} \theta(\bm{X},t) \equiv \mathbf{N}^{g}(\bm{X}) \bm{\uptheta}(t), \quad{ }^{h} \theta \in{ }^{h} \mathscr{K}_\theta.
\end{equation}
\end{highlight}

It is also convenient to defined the discrete gloabal gradient operator \(\mathbf H^g\).
For instance, in a 2D problem, where cartesian coordinates are employed, this discrete operator is defined as
\begin{equation} \label{eq:discrete_grad_op}
  \mathbf H^g\equiv \left[
  \begin{array}{cccc}
    \displaystyle{\frac{\partial N^g_1}{\partial X}} & \displaystyle{\frac{\partial N^g_2}{\partial X}} & \dots & \displaystyle{\frac{\partial N^g_{n_\text{points}}}{\partial X}} \\[10pt]
    \displaystyle{\frac{\partial N^g_1}{\partial Y}} & \displaystyle{\frac{\partial N^g_2}{\partial Y}} & \dots & \displaystyle{\frac{\partial N^g_{n_\text{points}}}{\partial Y}}
  \end{array}
  \right].
\end{equation}

Applying the aforementioned finite element discretization to the  thermal initial BVP yields
\begin{highlight}
\begin{equation}
  \int_{^h\Omega_0}   \left[\left(C_\mathrm{V} \dot \theta - \rho_0 r\right)  \mathbf N^g\bm \upxi +k\bm g_0\cdot \mathbf H^g \bm \upxi\right]\ud v - \int_{^h\partial\Omega_0} h_0 \mathbf N^g\bm \upxi\ud a = 0,\quad \forall \bm \upxi \in {}^h\mathscr{V}_{\theta,0},
\end{equation}
\end{highlight}
which can be rewritten
\begin{equation} \label{eq:thermo_variational_lemma}
  \left\{\int_{^h\Omega_0}   \left[(\mathbf N^g)^T\left(C_\mathrm{V} \dot \theta - \rho_0 r\right) +k(\mathbf H^g)^T\mathbf H^g \bm \uptheta \right]\ud v - \int_{^h\partial\Omega_0} (\mathbf N^g)^T h_0 \ud a \right\}^T\bm \upxi= 0,\quad \forall \bm \upxi \in {}^h\mathscr{V}_{\theta,0},
\end{equation}
where the relation \(\bm g_0 = \mathbf H^g \bm \uptheta\) is employed.
Since Equation~\eqref{eq:thermo_variational_lemma} must be satisfied for any \(\bm \upxi\in {}^h\mathscr V_{\theta,0}\), the discretized  thermal initial boundary value problem can be statted as
\begin{problem}[Discretized  thermal initial BVP.]
GGiven the prescribed heat sources and heat fluxes $r(\bm X, t)$ and $h_0(\bm X, t)$ find the admissible nodal temperatures $\theta(t)\in {^h\mathscr{K}_{\theta}}$ such that the body $\mathscr{B}$ is in energetic equilibrium
\begin{equation}
    \mathbf C \dot{ \bm\uptheta}(t) +\mathbf K\bm\uptheta(t)-\mathbf f^\text{\;ext}(t)=\mathbf 0, \label{eq:equilibrium_material}
\end{equation}
where $\mathbf C$ and \(\mathbf K\) are the temperature damping and stiffness matrix defined as
\begin{align}
  \mathbf C &= \int_{{}^h\Omega_0} C_\mathrm{V} {\mathbf{N}^g}^T \mathbf{N}^g \ud v.,\\
  \mathbf K &= \int_{{}^h\Omega_0} k {\mathbf{H}^g}^T\mathbf{H}^g \ud v.
\end{align}
and $\mathbf f^\text{\;ext}(t)$ is the global vector of external forces defined as
\begin{align}
    \mathbf f^\text{\;ext}(t) &\equiv \int_{^h\Omega_0} \rho{\mathbf N^g}^T r(\bm X, t)\ud v + \int_{\partial^h\Omega_\text{heat,0}}{\mathbf N^g}^T h_0(\bm X, t)\ud a.
\end{align}
\end{problem}
