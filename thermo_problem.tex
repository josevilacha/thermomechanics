\chapter{Thermo field}

For the development of the thermomechanical models, appropriate for the target application to rocket nozzles, the temperature field needs to be considered. This section provides an overview of the governing equations required to describe a temperature field with the finite element method (FEM). A more detailed representation on this topic can be found in the literature, e.g. in Holzapfel [58], Lemaitre and Chaboche [78], and Polifke and Kopitz [105]. The procedure to establish a fully discrete system of equations for the themal field is comparable to the one for the structural field in chapter 3. Moreover, the basics of nonlinear continuum thermodynamics have already been featured in chapter 2. Consequently, the detailed derivation are skipped in this chapter.

In a first step, the balance equations for the thermal field will be established. Then, in a second step the thermal initial boundary value problem (IBVP) will be presented followed by the numerical solution technique. Latter requires a weak form of the thermal balance equation which will be fully discretised using the FEM for space discretisation and the finite difference method for time discretisation. To finish, the residual and the tangential system matrix will be introduced to enable the application of a Newton-Raphson method.

\section{Governing equations}

Based on the general model presented in Section~\ref{}, the balance equations for the temperature field are obtained as special case by neglecting all mechanical terms.
Hence, the energy balance equation (Equation~\eqref{eq:first_principle_thermo}), now in material description, reduces to
\begin{equation}
\rho_0(\dot{\psi}+\dot{\theta} s+ \theta \dot{s})=-\operatorname{div}_0 \bm Q+\rho_0 r \quad \text { in } \Omega_0,
\end{equation}
where all mechanical terms are neglected and the rate of the internal energy \(\dot{e}\) is replaced using Equation~\eqref{eq:def_helmholtz_free_energy}.
The target application of the present thesis is on coupled generally nonlinear thermomechanical interaction problems, where the initial and the current domains are not equal, i.e. \(\Omega_{0} \neq \Omega\).
Thus, for the sake of simplicity and in view of the later coupled problem, all following relations are expressed in material quantities.
A purely thermal analysis is independent of the deformation, so that reference and current configuration are identical and the domain remains constant, i.e. \(\Omega_{0} \equiv \Omega\).
Consequently, for a purely thermal analysis, the equations can be treated similar to a geometrically linear analysis, where e.g. the deformation gradient \(\bm F\) reduces to the identity tensor \(\bm I\).

\section{Thermal constitutive initial value problem}

From Section~\ref{}, discarding all variables related to the mechanical problem, the general thermal constitutive initial value problem is
\begin{problem}[General thermal constitutive intial value problem.]
GGiven the initial value of the temperature \(\theta(t_0)\), find the function for $\theta(t)$ such that the constitutive equations
\begin{gather}
    s = -\frac{\partial \psi}{\partial \theta},\\
    \psi = \psi(\theta),\\
    \frac{1}{\theta}\bm q = g(\theta, \bm g).
\end{gather}
are satisfied for every $t\in [t_0, t_\text{end}]$.
\end{problem}
No distinction between spatial and material configurations applies as \(\Omega = \Omega_0\).


% As a next step, the general potential ( \(2.88\) ) is reduced to the thermal terms, resulting in
% \[
% \psi(T, \operatorname{Grad} T)
% \]
% where the potential \(\psi\) depends only on the temperatures \(T\) and the temperature gradients \(\operatorname{Grad} T\). To describe thermodynamically admissible processes, the second law of thermodynamics, for instance in form ( \(2.94\) ), neglecting again all mechanical terms, has to be satisfied. Subsequently, time derivation of (4.2) reads
% \[
% \dot{\psi}=\frac{\partial \psi}{\partial T} \dot{T}+\frac{\partial \psi}{\partial \operatorname{Grad} T} \cdot(\operatorname{Grad} T)^{\prime}
% \]
% Introducing (4.3) into the reduced form of ( \(2.94\) ) then yields
% \[
% -\rho_{0}\left(\frac{\partial \psi}{\partial T}+\eta\right) \dot{T}-\rho_{0} \frac{\partial \psi}{\partial \operatorname{Grad} T} \cdot(\operatorname{Grad} T)-\frac{1}{T} Q \cdot \operatorname{Grad} T \geq 0
% \]
% where the entropy \(\eta\) is defined by the first term of (4.4) or by ( \(2.99\) ). In contrast to the purely mechanical theory in chapter 3 , the entropy \(\eta\) cannot be neglected for the thermal field but represents a variable of the problem at hand. Moreover, the second term of \((4.4)\) is determined according to (2.100). Hence, the entropy inequality reduces to the heat conduction dissipation \(\mathcal{D}_{\text {cond }}\) introduced in its spatial version in (2.82). Accordingly, the material version follows as
% \[
% \mathcal{D}_{\text {cond }}=:-\frac{1}{T} Q \cdot \operatorname{Grad} T \geq 0
% \]

Next, a standard set of assumptions are introduced.
As a first step, the specific heat \(C_{V}\) is established and defined according to the thermodynamical principles to be the amount of heat required to change a unit mass of a substance by one degree in temperature, i.e.
\begin{equation}
C_{\mathrm{V}}=\frac{\partial e}{\partial \theta}.
\end{equation}
The index \((\cdot)_{v}\) denotes that \(C_{v}\) is measured at constant volume.
Its dimensions are energy over temperature, i.e., \(\mathrm{[E/\Theta]}\), and using the International System of Units (SI), \(C_{V}\) is expressed in joules per kelvin.
Using Equation~\ref{eq:def_helmholtz_free_energy}, the specific heat at constant volume follows can be written as
\begin{equation}
C_{\mathrm{V}}=-\frac{\partial^{2} \psi}{\partial \theta^{2}} \theta=\frac{\partial s}{\partial \theta} \theta.
\end{equation}

As an assumption, the Helmholtz free energy per unit reference volume\textcolor{red}{(?)} is chosen to be
\begin{highlight}
\begin{equation}
\psi(\theta)=- C_{\mathrm{V}}\left[\left(\theta-\theta_{0}\right)-\theta \ln \left(\frac{\theta}{\theta_{0}}\right)\right],
\end{equation}
\end{highlight}
where \(\theta_{0}\) and \(C_{\mathrm{V}}\) denote the constant initial temperature and the constant specific heat, respectively.
In general, the heat capacity depends on the deformation and on the temperature.
However, for the application to elastomers, see for instance Netz [96], the heat capacity \(C_{V}\) can be assumed to depend only on the temperature.
Furthermore, for the application to metals, a constant specific heat capacity (i.e. \(C_{\mathrm{V}}=\) const.) is a valid assumption, utilised e.g. in Adam and Ponthot [1], Ghadiani [48], Ibrahimbegovic and Chorfi [61], and Simo and Miehe [122].
Accordingly, the heat capacity is also assumed to be constant (i.e. \(C_{\mathrm{V}}= \mathrm{const}\).), since focus in this work is on the application to metals.
Subsequently, the entropy and its time derivative yield
\begin{align}
s(\theta)&=-\frac{\partial \psi}{\partial \theta}=C_{\mathrm{V}} \ln \left(\frac{\theta}{\theta_{0}}\right) \\
\dot{s}(\theta)&=\frac{\partial s}{\partial \theta} \dot{\theta}=-\frac{\partial^{2} \psi}{\partial \theta^{2}} \dot{\theta}=C_{\mathrm{V}} \frac{1}{\theta} \dot{\theta}.
\end{align}


As previously mentioned, in a purely thermal analy sis the deformation is neglected, consequently the material and spatial heat flux coincide, that is \(Q \equiv q\), which is also valid for the material and spatial gradient, hence \(\operatorname{Grad} T \equiv \operatorname{grad} T\). Subsequently, to satisfy (4.5), a constitutive law for the heat flux has to be chosen associating the heat flux \(q\) with its dual variable grad \(T\) and the temperature \(T\). Accordingly, so-called Fourier's law proposed by Fourier [41], which is linear and isotropic is utilised, which is defined as
\[
q=-k \operatorname{grad} T
\]
Herein, the thermal conductivity \(k\) is assumed constant and positive that is \(k \geq 0\). Thus, heat is conducted in the direction of decreasing temperatures. Apart from Fourier's law, different constitutive laws for the heat flux are available in the literature, as e.g. Duhamel's law of heat conduction (see e.g. Holzapfel [58]) which uses a positive semi-definite second-order tensor \(k\) instead of the constant conductivity \(k\). If Duhamel's law is restricted to thermally isotropic behaviour (i.e. no preferred direction), the conductivity tensor reduces to \(k=k \boldsymbol{I}\). If a constant heat conductivity \(k=\) const. is assumed, Fourier's law is recovered as a special form of Duhamel's law. Moreover, e.g. in Holzapfel and Simo [59] and Sherief and Abd El-Latief [117], a variable conductivity \((k \neq\) const. \()\) is assumed in the context of elastomers. In Bargmann and Steinmann [13] and Bargmann et al. [14], three different constitutive laws for the heat flux \(q\) are proposed based on the Green-Naghdi's non-classical theory. Nevertheless, for the present work Fourier's law as stated in (4.12) yields physical results and hence is exclusively considered in this thesis.

\begin{problem}["Standard" thermal constitutive intial value problem.]
GGiven the initial value of the temperature \(\theta(t_0)\equiv \theta_0\), find the function for $\theta(t)$ such that the constitutive equations
\begin{gather}
    s =C_{\mathrm{V}} \ln \left(\frac{\theta}{\theta_{0}}\right),\\
    \psi(\theta) = - C_{\mathrm{V}}\left[\left(\theta-\theta_{0}\right)-\theta \ln \left(\frac{\theta}{\theta_{0}}\right)\right],\\
    \bm q = -k\bm g.
\end{gather}
are satisfied for every $t\in [t_0, t_\text{end}]$.
\end{problem}

\begin{problem}[Weak energy balance equations.]
TThere is energy balance in the body if and only if the temperature distribution satisfies
    \begin{equation}
        \int_\Omega   \left[\left(\dot e - \rho r\right) \xi - \bm q\cdot \nabla \xi\right]\ud v - \int_{\partial\Omega} \bm q\cdot \bm n_0 \xi\ud a = 0,\quad \forall \vect \xi \in \mathscr{W},
    \end{equation}
 where $\mathscr{W}$ is the space of virtual temperature variations of the body, defined by the space of sufficiently regular arbitrary temperature variations.
 \end{problem}

\section{The thermal initial boundary value problem}

Following the same approach as in Section~\ref{}, it is now possible to introduce the the thermal initial boundary value problem.
Assume that the internal variables governing the body \(\mathcal B\) are known at the initial time \(t_0\).
In addition, assume that the heat generated in the interior of the body is prescribed, \(r(\bm x, t)\), \(t\in[t_0, t_\text{end}]\), as well as,
\begin{itemize}
  \item \textbf{Natural (or Neumann) boundary condition.} The boundary portion \(\partial \Omega_\text{heat}\) of \(\mathcal B\) is subject to a prescribed history of heat flow, \(\bm q_\text{presc}(\bm x, t)\), \(\bm x \in \partial \Omega_\text{heat}\), \(t\in [t_0,t_\text{end}]\).
  \item \textbf{Essential (or Dirichlet) boundary condition.} The boundary portion \(\partial \Omega_\text{temperature}\) of \(\mathcal B\) is subject to a prescribed temperature history, \(\theta_\text{presc}(\bm x, t)\), \(\bm x \in \partial \Omega_\text{temperature}\), \(t\in [t_0,t_\text{end}]\).
\end{itemize}

4.2 Finite element formulation and solution schemes
The IBVP of the thermal field is described by the equations (4.11) and (4.12) combined with the kinematic relations presented in section \(2.1\), as well as with a set of initial conditions and boundary conditions. The boundary \(\partial \Omega_{0}\) is divided into pairwise disjoint boundary parts \(\partial \Omega_{0}=\) \(\Gamma_{0 ; \mathrm{D} ; \mathrm{T}} \cup \Gamma_{0 ; \mathrm{N} ; \mathrm{T}}\) where the index \(\mathrm{T}\) represents the boundary of the thermo problem. Dirichlet and Neumann boundary conditions are prescribed on \(\Gamma_{0 ; \mathrm{D} ; \mathrm{T}}\) and \(\Gamma_{0 ; \mathrm{N} ; \mathrm{T}}\), respectively, as follows:
\[
\begin{aligned}
T=\hat{T} & \text { on } \Gamma_{0 ; \mathrm{D} ; \mathrm{T}} \\
-\boldsymbol{Q} \cdot \boldsymbol{n}_{0}=\hat{Q} & \text { on } \Gamma_{0 ; N ; T}
\end{aligned}
\]
Herein, \(\hat{Q}\) is defined as the heat flux in opposite or negative direction of the outward normal vector indicated by the negative value of \(-\boldsymbol{Q} \cdot \boldsymbol{n}_{0}\), i.e. inflow into the body is postulated to be positive. On a specific part \(\Gamma_{0 ; \text { C } ; T}\) of the Neumann boundary \(\Gamma_{0 ; N ; T}\), a heat flux according to Newton's law of heat dissipation, so-called heat convection boundary conditions, can be prescribed in the following form:
\[
-Q \cdot n_{0}=-(-k \operatorname{Grad} T) \cdot n_{0}=: \hat{Q}_{c}=h\left(T-T_{\infty}\right) \quad \text { on } \Gamma_{0 ; c ; \mathrm{T}}
\]
with linear heat transfer coefficient \(h\) and ambient temperature \(T_{\infty}\) of the surrounding. Given the initial temperature field \(T_{0}\), the initial condition at \(t=0\) reads
\[
T_{0}=T(\boldsymbol{X}, t=0)=\hat{T}_{0} \quad \text { in } \Omega_{0}
\]
A weak form of the instationary heat conduction equation is obtained by multiplication of (4.11) and (4.14) with the virtual temperatures \(\delta T\) followed by integration by parts as
\[
\begin{array}{l}
\int_{\Omega_{0}} \rho_{0} C_{V} \dot{T} \delta T \mathrm{~d} V_{0}-\int_{\Omega_{0}} Q \cdot \operatorname{Grad} \delta T \mathrm{~d} V_{0} \\
-\int_{\Gamma_{0 \text { N } ; T} \backslash \Gamma_{0, c ; \mathrm{T}}} \hat{Q} \delta T \mathrm{~d} A_{0}-\int_{\Gamma_{0 ;<; \mathrm{T}}} \hat{Q}_{\mathrm{C}} \delta T \mathrm{~d} A_{0}-\int_{\Omega_{0}} \rho_{0} r \delta T \mathrm{~d} V_{0}=0
\end{array}
\]
where the virtual temperatures \(\delta T\) are assumed to \(\delta T=0\) on \(\Gamma_{0 ; D ; T}\). In contrast to the weak form of the structural field (3.11) which describes virtual works \(\delta \mathcal{W}\), the weak form of the thermal field (4.17) describes the rate of virtual work, i.e. a virtual power \(\delta \mathcal{P}\).

As before the admissible temperature distributions for the body \(\mathcal B\) are all sufficiently regular temperature fields that satisfy the essential boundary condition,
\begin{equation}
  \mathscr K = \{\theta:\Omega \times \mathbb R \to \mathbb R\,|\,\theta(\bm x, t) = \theta_\text{presc}(\bm x, t),\quad \bm x\in \partial \Omega_\text{temperature},\quad t\in[t_0, t_\text{end}]\}.
\end{equation}

Including (4.2)-(4.10) into (4.1) and reformulating, postulates the strong form of the instationary heat equation via
\[
\rho_{0} C_{\mathrm{V}} \dot{T}=-\operatorname{Div} Q+\rho_{0} r \quad \text { in } \Omega_{0}
\]

Thus, the weak form of the "standard" thermal constitutive initial boundary value problem can be stated as follows
 \begin{problem}["Standard" thermal initial BVP.]
     FFind an admissible temperature distribution, $\theta \in \mathscr{K}$, such that for every $t\in [t_0,t_\text{end}]$, the body $\mathscr{B}$ is in energetic equilibrium
         \begin{equation}
         \int_\Omega   \left[\left(C_\mathrm{V} \dot \theta - \rho r\right) \xi +k\bm g\cdot \nabla \xi\right]\ud v - \int_{\partial\Omega} \bm q\cdot \bm n_0 \xi\ud a = 0,\quad \forall \vect \xi \in \mathscr{W},
     \end{equation}
     where the space of virtual temperature distributions at time $t$ is defined by
     \begin{equation}
         \mathscr{W} \equiv \left\{\xi:\Omega_0\to \mathbb R\;|\;\xi = 0\quad \text{in}\quad \partial\Omega_\text{temperature,0}\right\}.
     \end{equation}
 \end{problem}

\section{Time discretization}


 \begin{problem}[`Standard' incremental thermal initial BVP.]
     GGiven the temperature distribution $\theta_n$ at $t_n$, the prescribed heat sources and heat fluxes force $r_{n+1}$ and $\bm q_{n+1}$ at $t_{n+1}$, find the admissible temperature distribution $\theta_{n+1}\in\mathscr{K}_{n+1}$ such that the body $\mathscr{B}$ is in energetic equilibrium
             \begin{equation}
         \int_\Omega   \left[\left(C_\mathrm{V} \dot \theta_{n+1} - \rho r_{n+1}\right) \xi +k \bm g_{n+1}\cdot \nabla \xi\right]\ud v - \int_{\partial\Omega} \bm q_{n+1}\cdot \bm n_0 \xi\ud a = 0,\quad \forall \xi \in \mathscr{W},
     \end{equation}
     where the space of admissible temperature distributions $\mathscr{K}_{n+1}$ is defined by
     \begin{equation}
             \mathscr{K}_{n+1}\equiv \{\theta:\Omega\times \mathbb{R}\to \mathbb{R}\;|\;\theta_{n+1}(\bm x) = \theta_\text{presc,$n+1$}(\vect x),\;\vect x\in\partial \Omega_\text{temperature}\}.
     \end{equation}
 \end{problem}

\section{Finite Element Method}

By convenience, the global shape functions can be assembled in the so-called global interpolation matrix as
\begin{equation}
\mathbf{N}^{g}(\boldsymbol{x}) \equiv\left[N_1^g(\bm x), N_2^g(\bm x), \dots, N^g_{n_\text{points}}(\bm x)\right],
\end{equation}

\begin{equation}
 \bm \uptheta = \left[\theta^{1}, \dots, \theta^{n_{\text {points}}}\right]^{T}.
\end{equation}

\begin{highlight}
\begin{equation}
{ }^{h} \theta(\bm{x}) \equiv \mathbf{N}^{g}(\bm{x}) \bm{\uptheta}, \quad{ }^{h} \theta \in{ }^{h} \mathscr{K}.
\end{equation}
\end{highlight}

\begin{equation}
  \mathbf H^g\equiv \left[
  \begin{array}{cccc}
    \displaystyle{\frac{\partial N^g_1}{\partial x}} & \displaystyle{\frac{\partial N^g_2}{\partial x}} & \dots & \displaystyle{\frac{\partial N^g_{n_\text{points}}}{\partial x}} \\[10pt]
    \displaystyle{\frac{\partial N^g_1}{\partial y}} & \displaystyle{\frac{\partial N^g_2}{\partial y}} & \dots & \displaystyle{\frac{\partial N^g_{n_\text{points}}}{\partial y}}
  \end{array}
  \right].
\end{equation}

\begin{problem}[Material incremental discretized mechanical initial BVP.]
GGiven the temperature $\theta_n$ at $t_n$, the prescribed heat sources and heat fluxes $r_{n+1}$ and $\bm q_{n+1}$ find the admissible nodal temperatures $\theta_{n+1}\in {^h\mathscr{K}_{n+1}}$ such that the body $\mathscr{B}$ is in energetic equilibrium
\begin{equation}
    \mathbf C \dot{ \bm\uptheta}_{n+1} +\mathbf K\bm\uptheta_{n+1}-\mathbf f^\text{\;ext}_{n+1}=\mathbf 0, \label{eq:equilibrium_material}
\end{equation}
where $\mathbf C$ and \(\mathbf K\) are the temperature damping and stiffness matrix defined as
\begin{align}
  \mathbf C &= \int_{{}^h\Omega} C_\mathrm{V} {\mathbf{N}^g}^T \mathbf{N}^g \ud v.,\\
  \mathbf K &= \int_{{}^h\Omega} k {\mathbf{H}^g}^T\mathbf{H}^g \ud v.
\end{align}
and $\mathbf f^\text{\;ext}_{n+1}$ is the global vector of external forces defined as
\begin{align}
    \mathbf f^\text{ext}_{n+1} &\equiv \int_{^h\Omega} \rho{\mathbf N^g}^T r_{n+1}\ud v + \int_{\partial^h\Omega_\text{heat}}{\mathbf N^g}^T \mathbf q_{n+1}\cdot \mathbf n_0\ud a.
\end{align}
\end{problem}
