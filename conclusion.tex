\chapter{Conclusion and Future Works}

The present work presents a partitioned thermomechanical solver.

The current work focuses on the developement of an implicit partitioned thermomechanical solver.

It presents a comprehensive dissertation on the thermodynamically consistent continuum mechanics.
It follows with the strictly mechanical problem, the strictly thermal problem and the full thermomechanical problem.
The corresponding intial value problems for the constitutive problem are introduced, as is the weak formulation of the relevant conservation and balance principles and their spatial and temporal discretization.

It follows a validation of the thermal solver.
The mechanical solver is not validated as it is part of the LINKS code used as the basis for the current developments.
Appropriate references are used in DIN 1992 and the NAFEMS benchmarks.
There is a good agreement between the numerical results and the references.

It follows a thorough investigation on the available approaches to solve coupled problems, with a special focus on thermomechanical problems.
A large sweep of the literature is performed, with the main classes of solution procdures being the monolithic approaches and the partitioned approaches.
The partioned approaches can be further divided into loosely coupled or explicit and strongly coupled or implicit.
Given the requirements put forth the most promising solution is determined to be a strongly coupled or implicit partitioned scheme.

Having performed this choice the following step is to understand the implicit methods available.
Recasting the problem as a system of nonlinear equations, where the residual is the difference bewteen the initial input and its output after applying the fixed-point.
This approach leads to the consideration of a large family of methods for the solution of nonlinear equations.
These are presented in detail for the solution of coupled multi-physics systems.
These are the fixed-point method, the underrelaxation method, the Aitken relaxation, the Broyden-like family of methods, the Newton-Krylov methods and the polynomial vector extrapolation methods, MPE and RRE, in cycling mode.

The validation of the thermomechanical solver and the implicit solution methodsm, as well as their comparison, is performed using to examples, whose results are present in the literature.
The expansion of an infinitely long thick-walled thermoelastic cylinder, and the necking of circular thermoelastoplastic bar.
The numerical results agree with the references provided confidence in the solution developed.
Regarding the comparison of the different implicit techniques, the best performing are the Broyden-like methods with \(\beta=-1\), Type I update and \(s=1\), corresponding to the good Broyden method, and \(s=2\).
These are both computationally efficient with few calls to the residual function and not very memory intensive.
The Aitken relaxation being the simplest and the least memory intensive also performs well.
The other methods considered, including the Newton-GMRES and the MPE in cycling mode, display a worse performance.
There is however a caveat regarding the Newton-Krylov methods regarding the possible use of global strategies such as line search given the accurate estimate for the Jacobian of the residual.
Moreover, it has been determined that most computationally demanding portion of the implicit partitioned schemes is the solution of the mechanical and thermal problems, with the manipulation concerning solely the coupling solver taking a very minute portion of the total computational time.

\section{Future Works}

Future works include the used of global strategies creating a more robust procedure.
The inclusion of more methods for the solutin of nonlinear problems.
Experimentation with solving the monolithic scheme using the implicit methods presented which are Jacobian free.
The development of a thermomechanical constitutive model for a semi-crystalline polymer.
