\chapter{Thermo-mechanical problem}
\label{ch:thermo_mechanical_problem}

The full thermo-mechanical analysis is conducted in the subsequent chapter using the overall framework discussed in Chapter~\ref{ch:continuum_mechanics}.
The independent mechanical and thermal problems discussed in the Chapters~\ref{ch:mechanical_problem} and \ref{ch:thermal_problem} serve as a roadmap for the developments that come next.
Since the two fields are related at every point of the domain, this coupling is referred to as a volume coupling.
In contrast, surface-coupled issues, such as those involving fluid-structure interactions, encompass situations where coupling only occurs at the boundary between the fluid and the structural domain.
The thermomechanical potential is the typical methodology for finite deformation thermomechanical analysis \citep{danowski_computational_2014}.



\subsection{Thermo-mechanical constitutive initial value problem}

The following set of equations in the material configuration set up a constitutive model based on internal variables in the complete thermo-mechanical scenario, taking into account all quantities relevant to both the mechanical and the thermal domain
    \begin{gather}
        \mat P = \rho_0 \frac{\partial \psi}{\partial \mat F},\\
        s = - \frac{\partial \psi}{\partial \theta},\\
        \psi = \psi(\mat F,\theta, \mat \alpha),\label{eq:thermo_mech_helmholtz_free_energy}\\
        \dot{\bm \alpha} = f(\mat F, \theta, \bm g_0,\bm \alpha),\\
        \bm q_0 = g(\bm F, \theta, \bm g_0, \bm \alpha).
    \end{gather}

% It is already mentioned in Chapter~\ref{ch:continuum_mechanics}, that the Helmholtz free energy is permitted to vary with temperature (\theta\).
% So, the temperature is directly incorporated into the constitutive equation.
In addition to the plastic intermediate configuration that must be taken into account after the establishment of finite strain plasticity in the general non-isothermal situation, a thermal intermediate configuration must also be considered, i.e.,
\begin{equation}
  \bm F = \bm F^t\,\bm F^e\,\bm F^p.
\end{equation}

The volume is constant during plastic deformations for the majority of metals, unlike thermal deformations, which are only volumetric.
Hence, plastic deformations are assumed isochoric, \(J^p = \operatorname{det}\bm F^p=1\), yielding the following split for the jacobian of the deformation
\begin{equation}
  J = J^t J^e.
\end{equation}
Following \cite{danowski_computational_2014}, the thermal expansion is assumed to be
\begin{equation}
  J^t = \frac{\ud v}{\ud v_0} = \operatorname{exp}(3\alpha_T \Delta\theta).
\end{equation}
Combining the last two equations, the elastic volumetric deformation \(J^e\), can be expressed as
\begin{equation} \label{eq:jacobian_elasatic_temp}
  J^e  = J^e(\theta) = J/J^t.
\end{equation}
As a result, the additional thermal intermediate configuration can be dropped, leaving only the volumetric deformations to be considered.
Elastic strains balance the body, which implicitly match the thermal strains, according to Equation~\eqref{eq:jacobian_elasatic_temp}, if thermal stresses develop as a result of a temperature change.

The Helmholtz free energy \(\psi\) in Equation~\ref{eq:thermo_mech_helmholtz_free_energy} is expressed with respect to the reference volume, so that \(\psi\) is reformulated using potential functions,to emphasize this additive decomposition, as follows
  \begin{highlight}
    \begin{equation}
    \rho_{0} \psi\left(\bm{F}, \theta, \bm{\alpha}\right):=\hat{\mathbb{U}}\left(J^e,\theta\right)+\hat{\mathbb{W}}(\bm{F}_\text{iso},\theta)+\hat{\mathbb{M}}\left(J^e, \theta\right)+\hat{\mathbb{T}}(\theta)+\hat{\mathbb{K}}\left(\boldsymbol{\alpha}, \theta\right),
    \end{equation}
  \end{highlight}
  where in contrast to the deformation gradient \(\bm{F}\), the Jacobi-determinant \(J^{e}\) relative only to the elastic component and the isochoric deformation gradient \(\bm{F}_\text{iso}\) are applied.
  \(\hat{\mathbb U}\) and \(\hat{\mathbb{W}}\) can be identified with the standard hyperelastic materials potentials, with the caveat that now the material parameters can depend on the temperature, whereas
  \(\hat{\mathbb M}\) describes the thermomechanical coupling potential.
  The potential \(\hat{\mathbb{T}}\) represents the purely thermal potential.
  Finally, \(\hat{\mathbb{K}}\) is the convex plastic potential.

Based on the potential functions, the coupling of the two fields, mechanical and thermal, can be explained:
\begin{itemize}
  \item the temperature enters the structural field via additional thermal stresses and possibly via temperature-dependent material parameters.
  Herein, \(\hat{\mathbb{M}}\) characterizes the thermomechanical coupling potential, leading to thermal stresses and thermal expansion and dilatation, whereas \(\hat{\mathbb{K}}\) being temperature-dependent and therefore enables exemplarily von Mises plasticity combined with temperature-dependent isotropic hardening and thermal softening.
  \item the structure enters the thermal field via coupling terms, arising from \(\hat{\mathbb{M}}\) and \(\hat{\mathbb{K}}\), and thus, coupling terms as the internal dissipation \(\pazocal{D}_{\text {int}}\) may emerge in the thermal balance equation.
  Furthermore, for finite deformation thermomechanical problems, where the initial domain \(\Omega_{0}\) deforms to \(\Omega\), so that \(\Omega \neq \Omega_{0}\), and a Lagrangian formulation is used, the deformation enters the thermal field additionally due to the mapping of all quantities in the balance equations to the reference configuration.
\end{itemize}

As in Chapter~\ref{ch:thermal_problem}, the heat conduction law is chosen to be the Fourier heat conduction law, repeated here for the sake of clarity in the spatial description
\begin{equation}
  \bm q = - k \bm g.
\end{equation}
Applying the Piola transformation yields for the material description of the heat flux vector,
\begin{highlight}
  \begin{equation}
    \bm q_0 = - k_0 \bm C^{-1} \bm g_0.
  \end{equation}
\end{highlight}

Thus, the material thermo-mechanical constitutive initial value problem can be stated as
    \begin{problem}[Material thermomechanical constitutive initial value problem.]
    GGiven the initial values of the internal variables, $\bm \alpha(t_0)$, the history of the deformation gradient
    \begin{equation}
        \mat F(t),\quad t\in[t_0,t_\text{end}],
    \end{equation}
    and the history of the temperature distribution and its gradient
    \begin{gather}
    \theta(t),\quad t\in[t_0,t_\text{end}],\\
    \bm g_0(t), \quad t\in[t_0, t_\text{end}],
    \end{gather}
    find the functions for $\mat P(t)$, \(\bm q_0(t)\) and $\bm \alpha(t)$ such that the constitutive equations
    \begin{gather}
        \mat P = \rho_0 \frac{\partial \psi}{\partial \mat F},\\
        s = -\frac{\partial \psi}{\partial \theta},\\
        \psi =\frac{1}{\rho_0}\left( \hat{\mathbb{U}}\left(J^{e}\right)+\hat{\mathbb{W}}(\bm F_\text{iso})+\hat{\mathbb{M}}\left(J^{e}, \theta\right)+\hat{\mathbb{T}}(\theta)+\hat{\mathbb{K}}\left(\boldsymbol{\alpha}, \theta\right)\right),\\
        \bm q_0 = - k_0 \bm C^{-1} \bm g_0,\\
        \dot{\bm \alpha} = f(\mat F, \theta, \bm g_0 ,\bm \alpha),
    \end{gather}
    are satisfied for every $t\in [t_0, t_\text{end}]$.
    \end{problem}

\subsection{Weak equilibrium. The principle of virtual work}

The strong equations that enforce the equilibrium of a body can be writen using the material description as
\begin{align}
  \rho_0 \ddot{\bm u} = \operatorname{div}_0\bm P + \bm b_0, \quad & \text{in $\Omega_0$},\\
   \rho_0 \dot e = \bm P :\dot{\bm F} + \rho_0 r -\operatorname{div}_0 \bm q_0, \quad & \text{in  $\Omega_0$}.
\end{align}
Following an approach entirely similar to the ones presented in Chapters~\ref{ch:mechanical_problem} and \ref{ch:thermal_problem}, the weak form of the linear momentum and energy balance equations can be found to be
\begin{problem}[Weak form of the linear momentum and energy balance equations (material version).]
IIn a material description, the body is in mechanical equilibrium and energetic balance if and only if the First Piola-Kirchhoff stress field, \(\bm P(t)\), the heat flux vector \(\bm q_0(t)\), satisfy
    \begin{gather}
        \int_{\Omega_0} [\mat P(t):\nabla_0 \bm \eta - (\bm b_0(t) - \rho_0 \ddot{\bm u}(t))\cdot \bm \eta]\ud v - \int_{\partial\Omega_0} \bm t_0(t)\cdot \bm \eta\ud a = 0,\quad \forall \bm \eta \in \mathscr{V}_{u,0},\\
          \int_{\Omega_0}   \left[\left(\rho_0\dot e (t)- \bm P(t):\dot{\bm F}(t)- \rho_0r(t)\right) \xi - \bm q_0(t)\cdot \nabla_0 \xi\right]\ud v - \int_{\partial\Omega_0} h_0(t) \xi\ud a = 0,\quad \forall \xi \in \mathscr{V}_{\theta,0},
    \end{gather}
 where $\mathscr{V}_{u,0}$ is the space of virtual displacement of the body, defined by the space of sufficiently regular arbitrary displacements
 \begin{equation}
     \bm \eta\colon \Omega_0\to \mathscr{U}.
 \end{equation}
 and  $\mathscr{V}_{\theta,0}$ is the space of virtual temperature distributions of the body, defined by the space of sufficiently regular arbitrary temperature distributions
 \begin{equation}
     \xi\colon \Omega_0\to \mathscr R.
 \end{equation}
\end{problem}

\subsection{Mechanical constitutive initial boundary value problem}

It is now possible to pose the thermo-mechanical constitutive initial value problem in its weak form.
Assume that a body $\mathscr{B}$ is made from a generic material, characterized by a given constitutive model, whose internal variables are known at the initial time.
In addition, it is assumed that the interior of the body was subjected to a prescribed history of body forces, $\bm b(\bm X, t)$, and a prescribed history of heat sources, $r(\bm X,t)$, $t\in[t_0, t_\text{end}]$, and to the following boundary conditions:
\begin{itemize}
    \item \textbf{Natural (or Neumann) boundary condition:}
    \begin{itemize}
      \item \textbf{Mechanical field:} The boundary portion $\Omega_\text{traction, 0}$ of $\mathscr{B}$ is subjected to a prescribed history of traction forces, $\bm t_\text{presc,0}(\bm X, t)$, $\bm X\in \partial \Omega_\text{traction,0}$, $t\in[t_0, t_\text{end}]$,\\
      \item \textbf{Thermal field:} The boundary portion \(\partial \Omega_\text{heat,0}\) of \(\mathcal B\) is subject to a prescribed history of heat flux, \(h_\text{presc,0}(\bm X, t) = \bm q_\text{presc,0}(\bm X, t)\cdot \bm m(\bm X)\), \(\bm X \in \partial \Omega_\text{heat,0}\), \(t\in [t_0,t_\text{end}]\).
    \end{itemize}
    \item \textbf{Essential (or Dirichlet) boundary condition:}
    \begin{itemize}
      \item \textbf{Mechanical field:} The boundary portion $\Omega_\text{motion, 0}$ of $\mathscr{B}$ is subjected to a prescribed displacement field history, $\bm u_\text{presc}(\bm X, t)$, such that $$\bm \varphi(\bm X, t) = \bm X + \bm u_\text{presc}(\bm X, t),\quad \bm X\in \partial\Omega_\text{motion, 0},\quad t\in[t_0, t_\text{end}].$$
      \item \textbf{Thermal field:} The boundary portion \(\partial \Omega_\text{temperature,0}\) of \(\mathcal B\) is subject to a prescribed temperature history, \(\theta_\text{presc}(\bm X, t)\), \(\bm X \in \partial \Omega_\text{temperature,0}\), \(t\in [t_0,t_\text{end}]\).
    \end{itemize}
\end{itemize}

It is also convenient to define the set of kinematically admissible displacements and admissible temperature distributions of $\mathscr{B}$ as the set of all sufficiently regular displacement and temperature functions that satisfy the corresponding essential boundary conditions \citep{de_souza_neto_computational_2008},
\begin{highlight}[innertopmargin=-5pt]
    \begin{multline}
        \mathscr{K}_u\equiv \{\bm u:\Omega_0\times \mathscr{R}\to \mathscr{U}\;|\;\bm u(\bm X,t) = \bm u_\text{presc} (\bm X,t),\\ \bm X\in\partial \Omega_\text{motion,0},\quad t\in [t_0,t_\text{end}]\}.\quad
    \end{multline}
    \begin{multline}
        \mathscr{K}_\theta\equiv \{\theta:\Omega_0\times \mathscr{R}\to \mathscr{R}\;|\;\theta(\bm X,t) = \theta_\text{presc} (\bm X,t),\\ \bm X\in\partial \Omega_\text{temperature,0},\quad t\in [t_0,t_\text{end}]\}.\quad
    \end{multline}
\end{highlight}

Using the definition of Helmholtz's energy (Equation~\ref{eq:def_helmholtz_free_energy}), the balance of energy (Equation~\eqref{eq:strong_energy_eq}) can be written as
\begin{equation}
  \rho_0\dot \psi + \rho_0 \dot \theta s + \rho_0 \theta \dot s = \bm P:\dot{\bm F} + \rho_0 r - \operatorname{div}\bm q_0.
\end{equation}
Applying the chain rule to \(\dot s\) yields
\begin{equation}
  \dot s = \frac{\partial s}{\partial \theta}\dot \theta + \frac{\partial s}{\partial J^e }: \dot{J^e} + \frac{\partial s}{\partial \bm F_\text{iso}}: \dot{\bm F}_\text{iso} + \frac{\partial s}{\partial \bm \alpha}*\dot{\bm \alpha},
\end{equation}
Applying Equation~\eqref{eq:def_cv_partial} regarding the specific heat at constant volume and defining \(\pazocal H^\text{ep}\) as
\begin{equation}
    \pazocal H^\text{ep} = - \rho_0\theta\left(\frac{\partial^2 \hat{\mathbb U}}{\partial J^e\partial \theta}: \dot{J^e} + \frac{\partial^2 \hat{\mathbb W}}{\partial \bm{F}_\text{iso}\partial \theta}: \dot{\bm{F}}_\text{iso} + \frac{\partial^2 \hat{\mathbb M}}{\partial J^e\partial \theta}: \dot{J^e} + \frac{\partial^2 \hat{\mathbb K}}{\partial \bm \alpha \partial \theta}*\dot{\bm \alpha} \right)
\end{equation}
reveals
\begin{equation}
  \rho_0 (\dot \psi +  \dot \theta s) - \bm P:\dot{\bm F} + \rho_0 C_V\dot \theta - \pazocal H^\text{ep} = \rho_0 r - \operatorname{div}\bm q_0.
\end{equation}
The quantitiy denoted by \(\pazocal H^\text{ep}\) is the so-called Gough-Joule effect or thermo-elastoplastic heating (or cooling) effect.
According to Equation~\eqref{eq:def_int_dissipation}, the term \(\bm P:\dot{\bm F} - \rho_0(\dot \psi + \dot \theta s)\) is precisely the internal dissipation \(\pazocal D_\text{int}\).
Accounting also for the constitutive law chosen for the heat flux, i.e., Fourier's law (Equation~\eqref{eq:def_fourier_law}), one finally obtains
\begin{equation}
  \rho_0 C_V \dot \theta = \rho_0 r + k\operatorname{div} \bm g_0 + \pazocal D_\text{int} + \pazocal H^\text{ep}.
\end{equation}

So the weak form of the thermo-mechanical constitutive initial boundary value problem can be stated in a spatial description as follows
and in the material description as
\begin{problem}[Material thermo-mechanical initial BVP.]
    FFind a kinematically admissible displacement function, $\bm u\in \mathscr{K}_u$, and an admissible temperature distribution, \(\theta \in \mathscr K_\theta\), such that for every $t\in [t_0,t_\text{end}]$, the body $\mathscr{B}$ is in mechanical and energetic equilibrium
        \begin{equation}
        \int_{\Omega_0} [\mat P(t):\nabla_0 \bm \eta - (\bm b_0(t)-\rho_0\ddot{\bm u}(t))\cdot \bm \eta]\ud v - \int_{\partial\Omega_0} \bm t_0(t)\cdot \bm \eta\ud a = 0,\quad \forall \bm \eta \in \mathscr{V}_{u,0},
        \end{equation}
        \begin{multline}
        \int_{\Omega_0}   \left[\left(\rho_0C_\mathrm{V}\dot\theta(t) - \pazocal D_\text{int}(t) - \pazocal H^\text{ep}(t) - \rho_0 r(t)\right) \xi + k \bm g_0(t)\cdot \nabla_0 \xi\right]\ud v\\ - \int_{\partial\Omega_0} h_0(t) \xi\ud a = 0,\quad \forall \xi \in \mathscr{V}_{\theta,0},
    \end{multline}
    where the space of virtual displacements at time $t$ is defined by
    \begin{equation}
        \mathscr{V}_{u,0} \equiv \left\{\bm \eta:\Omega_0\to \mathscr{U}\;|\;\bm \eta = \bm 0\quad \text{in}\quad \partial\Omega_\text{motion,0}\right\},
    \end{equation}
    the space of virtual temperatures at time $t$ is defined by
    \begin{equation}
    \mathscr{V}_{\theta,0} \equiv \left\{\xi:\Omega_0\to \mathscr{R}\;|\; \xi =  0\quad \text{in}\quad \partial\Omega_\text{temperature,0}\right\},
    \end{equation}
    and at each point of $\mathscr{B}$, the First Piola-Kirchhoff stress tensor is the solution of material thermomechanical constitutive initial value problem.
\end{problem}

\section{Time descretization} \label{sec:time_discretization}

In the context of thermo-mechanical problems, a general path-dependent model is one that considers both the instantaneous deformation and temperature states as well as their history.
Under these circumstances, for complex strain, $\bm F(t)$, or temperature paths, $\theta(t)$, the solution of the constitutive initial value problem for a given set of initial conditions is typically unknown.
Therefore, it is necessary to employ a suitable numerical approach to integrate the rate constitutive equations, just as it would be in the case of a strictly mechanical problem.

Attending to the thermo-mechanical constitutive initial boundary value problem and considering the time increment $[t_n, t_{n+1}]$, this approach is comprised by the following requirements:
\begin{itemize}
    \item \textbf{First Piola-Kirchhoff stress tensors.} Considering a time increment $[t_n, t_{n+1}]$ and given the set $\bm \alpha_n$ of internal variables at $t_n$, the deformation gradient $\mat F_{n+1}$ and the temperature distribution \(\theta_{n+1}\) at time $t_{n+1}$ determines the First Piola-Kirchhoff stress tensor $\mat P_{n+1}$ uniquely as
    \begin{highlight}
      \begin{equation}
          \mat P_{n+1} = \hat{\mat P}(\mat F_{n+1}, \theta_{n+1}, \mat \alpha_n),
      \end{equation}
    \end{highlight}
    where $\hat{\mat P}$ is the incremental constitutive function for the First Piola-Kirchhoff stress tensor.
    \item \textbf{Set of internal variables.} Assuming that the set of internal variables $\bm \alpha_n$ is known at $t_n$, the set of internal variables must be uniquely determined by the prescribed deformation gradient $\mat F_{n+1}$ and temperature distribution \(\theta_{n+1}\) prescribed at $t_{n+1}$ as
    \begin{highlight}
        \begin{equation}
             \bm \alpha_{n+1} =\hat{\bm \alpha}(\mat F_{n+1}, \theta_{n+1}, \mat \alpha_n), \label{eq:incremental_flux}
        \end{equation}
    \end{highlight}
    where $\hat{\bm \alpha}$ is the incremental constitutive function for the set of internal variables.

    \item \textbf{Mechanical dissipation.}
 Considering a time increment $[t_n, t_{n+1}]$ and given the set $\bm \alpha_n$ of internal variables at $t_n$, the deformation gradient $\mat F_{n+1}$ and the temperature distribution \(\theta_{n+1}\) at time $t_{n+1}$ determines the mechanical dissipation \(\pazocal D_\text{int}\) as
    \begin{highlight}
    \begin{equation}
    \pazocal D_{\text{int},n+1} = \hat{\pazocal D}_\text{int}(\bm F_{n+1}, \theta_{n+1}, \bm \alpha_n).
    \end{equation}
    \end{highlight}

    \item \textbf{Gough-Joule effect.}
 Considering a time increment $[t_n, t_{n+1}]$ and given the set $\bm \alpha_n$ of internal variables at $t_n$, the deformation gradient $\mat F_{n+1}$ and the temperature distribution \(\theta_{n+1}\) at time $t_{n+1}$ determines the Gough-Joule effect \(\pazocal H^\text{ep}\) as
    \begin{highlight}
    \begin{equation}
    \pazocal H^\text{ep}_{n+1} = \hat{\pazocal H}^\text{ep}(\bm F_{n+1}, \theta_{n+1}, \bm \alpha_n).
    \end{equation}
    \end{highlight}
\end{itemize}
Generally, the numerical constitutive laws are nonlinear and path-independent within one increment.

Making use of the aforementioned time discretization, one can state the weak form of the mechanical constitutive initial boundary value problem in the material description as
\begin{problem}[Material incremental thermo-mechanical initial BVP.]
    GGiven the set of internal variables $\bm \alpha_n$ at $t_n$, the prescribed body and traction force fields $\bm b_{0,n+1}$ and $\bm t_{0,n+1}$, as well as, the prescribed heat sources, \(\bm r_{n+1}\) and heat fluxes, \(h_{0,n+1}\), at $t_{n+1}$, and the prescribed deformation gradient $\mat F_{n+1}$ and temperature distribution \(\theta_{n+1}\) at $t_{n+1}$, find the kinematically admissible displacement field $\bm u_{n+1}\in\mathscr{K}_{u,n+1}$ and the admissible temperature distribution \(\theta_{n+1}\in\mathscr K_{\theta,n+1}\) such that the body $\mathscr{B}$ is in mechanical and energetic equilibrium
            \begin{multline}
        \int_{\Omega_0} [\hat{\mat P}(\mat F(\bm u_{n+1}), \theta_{n+1}, \bm \alpha_n):\nabla_0 \bm \eta - (\bm b_{0,n+1}-\rho_0\ddot{\bm u}_{n+1})\cdot \bm \eta]\ud v\\ - \int_{\partial\Omega_0} \bm t_{0,n+1}\cdot \bm \eta\ud a = 0,\quad \forall \bm \eta \in \mathscr{V}_{u,n+1},
        \end{multline}
        \begin{multline}
        \int_{\Omega_0}   \left[\left(\rho_0C_\mathrm{V}\dot\theta_{n+1} -\hat{\pazocal D}_\text{int}(\bm F(\bm u_{n+1}), \theta_{n+1}, \bm \alpha_n) - \hat{\pazocal H}^\text{ep}(\bm{F}(\bm u_{n+1}), \theta_{n+1}, \bm \alpha_n)- \rho_0 r_{n+1} \right) \xi\right.\\
        \left. - \bm g_{0,n+1}\cdot \nabla_0 \xi\right]\ud v - \int_{\partial\Omega_0} \hat h_{0,n+1} \xi\ud a = 0,\quad \forall \xi \in \mathscr{V}_{\theta,n+1},
    \end{multline}
    where
    \begin{gather}
            \mathscr{K}_{u,n+1}\equiv \{\bm u:\Omega\times \mathscr{R}\to \mathscr{U}\;|\;\bm u_{n+1}(\bm X) = \bm u_\text{presc,$n+1$}(\bm X),\;\bm X\in\partial \Omega_\text{motion,0}\},\\
            \mathscr{K}_{\theta, n+1}\equiv \{\bm u:\Omega\times \mathscr{R}\to \mathscr{R}\;|\;\theta_{n+1}(\bm X) = \theta_\text{presc,$n+1$}(\bm X),\;\bm X\in\partial \Omega_\text{temperature,0}\}.
    \end{gather}
\end{problem}

\section{Finite Element Method} \label{sec:fem_mech}

It is now possible to apply the finite element method to discretize spatial the incremental thermo-mechanical initial boundary value problem.
The approach is entirely similar to the one present in the context of the strictly mechanical (see Section~\ref{sec:fem_mech}) and strictly thermal (see Section~\ref{sec:fem_therm}) problems.
As such, some details are omitted.

\subsection{Interpolation}

Defining the global vector of nodal displacements \(\mathbf u\) and the global vector of nodal temperatures \(\bm \uptheta\), like before, as
\begin{gather}
    \mathbf u(t) = \Big[ u_1^1(t),\dots,u^1_{n_\text{dim}}(t),\dots, u_1^{n_\text{points}}(t),\dots,u^{n_\text{points}}_{n_\text{dim}}(t)\Big]^T,\\
    \bm \uptheta(t) = \left[ \theta^1(t), \theta^2(t), \dots, \theta^{n_\text{points}}(t)\right]^T,
\end{gather}
the displacement $\bm u(\bm X, t)$ and temperature \(\theta(\bm X, t)\) fields defined over the global domain $\Omega_0$, can be found at any point $\bm X$ as
\begin{highlight}[innertopmargin=-5pt]
    \begin{gather}
        ^h\bm u(\bm X, t) \equiv \mathbf N^{g,u}(\bm X)\mathbf u(t),\quad ^h\bm u\in {}^h\mathscr{K}_u,\\
        ^h\theta(\bm X, t) \equiv \mathbf N^{g,\theta}(\bm X)\bm \uptheta(t),\quad ^h\bm \theta\in {}^h\mathscr{K}_\theta,
    \end{gather}
\end{highlight}
where \(\mathbf N^{g,u}\) and \(\mathbf N^{g,\theta}\) are the global interpolation matrices with appropriate dimensions given the dimensions of \(\mathbf u\) and \(\bm \uptheta\).
The global shape functions used to interpolate between the nodal values of the displacement and the temperature can even be different if need be.

\subsection{Spatial discretization} \label{sec:spatial_discretization}

The application of finite element discretization to the mechanical part of the incremental thermo-mechanical constitutive initial boundary value problem is exactly the same as the one presented in Section~\ref{sec:spatial_discretization_mech} for the strictly mechanical problem, as the the equation to be discretized is the same
On the other hand, applying the discretization to the thermal part of the incremental thermo-mechanical constitutive initial boundary value problem, yields
\begin{highlight}[innertopmargin=-5pt]
    \begin{multline}
    \int_{^h\Omega_0}   \left[\left(\rho_0C_\mathrm{V}\mathbf N^{g,\theta} \dot{\bm\uptheta}_{n+1} -\hat{\pazocal D}_\text{int} - \hat{\pazocal H}^\text{ep}- \rho_0 r_{n+1} \right) \cdot \mathbf N^{g,\theta}\bm{\upxi} - \mathbf H^{g,\theta} {\bm\uptheta}_{n+1} \cdot \mathbf H^{g,\theta} \bm{\upxi}\right]\ud v \\- \int_{^h\partial\Omega_0} h_{0,n+1} \mathbf N^{g,\theta}\bm{\upxi}\ud a = 0,\quad \forall \bm{\upxi} \in {}^h\mathscr{V}_{\theta}, \label{eq:thermo_mech_thermo_fem_disc}
    \end{multline}
\end{highlight}
where $\mathbf H^{g,\theta}$ is the discrete global gradient operator for scalars, defined in Equation~\eqref{eq:discrete_grad_op}.

Equation \eqref{eq:thermo_mech_thermo_fem_disc} can be rewritten as
\begin{multline}
  \left\{\int_{^h\Omega_0}   \left[{\mathbf N^{g,\theta}}^T \left(\rho_0C_\mathrm{V}\mathbf N^{g,\theta}\dot{\bm \uptheta}_{n+1} -\hat{\pazocal D}_\text{int} - \hat{\pazocal H}^\text{ep}- \rho_0 r_{n+1} \right) - {\mathbf H^{g,\theta}}^T\mathbf H^{g,\theta} {\bm\uptheta}_{n+1} \right]\ud v\right. \\ \left.- \int_{\partial{}^h\Omega_0} {\mathbf N^{g,\theta}}^T h_{0,n+1}\ud a\right\}^T \bm \upxi= 0,\quad \forall \bm{\upxi} \in {}^h\mathscr{V}_{\theta}, \label{eq:thermo_mech_thermo_fem_disc}
\end{multline}
and, since it must be satisfied for any $\bm \upxi \in {}^h \mathscr{V}_\theta$, the incremental discretized thermo-mechanical constitutive initial boundary value problem can thus be stated in the material description as
\begin{problem}[Material incremental discretized thermo-mechanical initial BVP.]
GGiven the set of internal variables $\bm \alpha_n$ at $t_n$, the prescribed body and traction force fields $\bm b_{0,n+1}$ and $\bm t_{0,n+1}$, as well as, the prescribed heat sources and heat flux fields \(\bm r_{0,n+1}\) and \(h_{0,n+1}\), and both the prescribed deformation gradient $\mat F_{n+1}$ and the prescribed temperature \(\theta_{n+1}\) at $t_{n+1}$, find the kinematically admissible nodal displacement field $\bm u_{n+1}\in {^h\mathscr{K}_{u,n+1}}$ and the admissible nodal temperature field \(\theta_{n+1}\in{^h\mathscr{K}_{\theta,n+1}}\) such that the body $\mathscr{B}$ is in mechanical and energetic equilibrium
\begin{gather}
    \mathbf M \ddot{\mathbf u}_{n+1} +\mathbf f_u^\text{\;int}(\bm \uptheta_{n+1}, \mathbf u_{n+1})-\mathbf f^\text{\;ext}_{u,n+1}=\mathbf 0, \\
    \mathbf C \dot{\bm \uptheta}_{n+1} + \mathbf K {\bm \uptheta}_{n+1} + \mathbf f_\theta^\text{\;int}(\bm \uptheta_{n+1}, \mathbf u_{n+1})-\mathbf f^\text{\;ext}_{\theta,n+1}=\mathbf 0,
\end{gather}
where $\mathbf f_u^\text{\;int}$ e $\mathbf f^\text{\;ext}_{u,n+1}$ are the mechanical global vectors of internal and external forces defined as
\begin{align}
    \mathbf f_u^\text{\;int} &\equiv \int_{^h\Omega_0}{\mathbf G^{g,u}}^T \hat{\mat P}(\mat F(\bm u_{n+1}), \theta_{n+1} \bm \alpha_n)\ud v,\\
    \mathbf f^\text{\;ext}_{u,n+1} &\equiv \int_{^h\Omega_0}{\mathbf N^{g,u}}^T \mathbf b_{0,n+1}\ud v + \int_{\partial^h\Omega_\text{traction,0}}{\mathbf N^{g,u}}^T \mathbf t_{0,n+1}\ud a,
\end{align}
and $\mathbf f_\theta^\text{\;int}$ e $\mathbf f^\text{\;ext}_{\theta,n+1}$ the thermal global vectors of internal and external forces defined as
\begin{multline}
    \mathbf f_\theta^\text{\;int} \equiv \left[\int_{^h\Omega_0}{\mathbf N^{g,\theta}}^T \hat{\pazocal D}_\text{int}(\mat F(\bm u_{n+1}), \theta_{n+1} \bm \alpha_n) +{\mathbf N^{g,\theta}}^T\hat{\pazocal H}^\text{ep}(\mat F(\bm u_{n+1}), \theta_{n+1} \bm \alpha_n)\right] \ud v,
\end{multline}
\begin{equation}
    \mathbf f^\text{\;ext}_{\theta,n+1} \equiv \int_{^h\Omega_0}{\mathbf N^{g,\theta}}^T \mathbf r_{0,n+1}\ud v + \int_{\partial^h\Omega_\text{heat,0}}{\mathbf N^{g,\theta}}^T h_{0,n+1}\ud a.
\end{equation}
The mass matrix $\mathbf M$ and the thermal capacitance matrix \(\mathbf C\) are defined by
\begin{align}
  \mathbf M &= \int_{{}^h\Omega_0} \rho_0 {\mathbf{N}^{g,u}}^T\mathbf{N}^{g,u} \ud v,\\
  \mathbf C &= \int_{{}^h\Omega_0} \rho_0C_\mathrm{V} {\mathbf{N}^{g,\theta}}^T\mathbf{N}^{g,\theta} \ud v,\\
  \mathbf K &= \int_{{}^h\Omega_0} {\mathbf H^{g,\theta}}^T\mathbf H^{g,\theta} \ud v.
\end{align}
\end{problem}

\pagebreak

The global vectors for the internal and external forces are usually obtained by assemblage of their elemental counterparts as
\begin{gather}
    \mathbf f_u^\text{\;int} = \assemble_{e=1}^{n_\text{elem}} \left(\mathbf f_u^\text{\;int}\right)^{(e)},\\
    \mathbf f_u^\text{\;ext} = \assemble_{e=1}^{n_\text{elem}} \left(\mathbf f_u^\text{\;ext}\right)^{(e)},\\
    \mathbf f_\theta^\text{\;int} = \assemble_{e=1}^{n_\text{elem}} \left(\mathbf f_\theta^\text{\;int}\right)^{(e)},\\
    \mathbf f_\theta^\text{\;ext} = \assemble_{e=1}^{n_\text{elem}} \left(\mathbf f_\theta^\text{\;ext}\right)^{(e)},
\end{gather}
where the mechanical elemental vectors in the material description are defined as
\begin{highlight}[innertopmargin=-5pt]
    \begin{align}
        \mathbf f_u^\text{\;int} &\equiv \int_{^h\Omega_0^{(e)}}{\mathbf G^{u}}^T \hat{\mat P}(\mat F(\bm u_{n+1}), \theta_{n+1} \bm \alpha_n)\ud v,\\
        \mathbf f^\text{\;ext}_{u,n+1} &\equiv \int_{^h\Omega_0^{(e)}}{\mathbf N^{u}}^T \mathbf b_{0,n+1}\ud v + \int_{\partial^h\Omega_\text{traction,0}^{(e)}}{\mathbf N^{u}}^T \mathbf t_{0,n+1}\ud a,
    \end{align}
\end{highlight}
and the thermal vectors, also in the material description, are
\begin{highlight}[innertopmargin=-5pt]
\begin{multline}
    \mathbf f_\theta^\text{\;int} \equiv \left[\int_{^h\Omega_0^{(e)}}{\mathbf N^{\theta}}^T \hat{\pazocal D}_\text{int}(\mat F(\bm u_{n+1}), \theta_{n+1} \bm \alpha_n) +{\mathbf N^{\theta}}^T\hat{\pazocal H}^\text{ep}(\mat F(\bm u_{n+1}), \theta_{n+1} \bm \alpha_n)\right] \ud v,
\end{multline}
\begin{equation}
    \mathbf f^\text{\;ext}_{\theta,n+1} \equiv \int_{^h\Omega_0^{(e)}}{\mathbf N^{\theta}}^T \mathbf r_{0,n+1}\ud v + \int_{\partial^h\Omega_\text{heat,0}^{(e)}}{\mathbf N^{\theta}}^T h_{0,n+1}\ud a.
\end{equation}
\end{highlight}
The elemental mechanical interpolation of the matrices $\mathbf N^u$, $\mathbf B^u$, \(\mathbf N^\theta\) and $\mathbf H^\theta$ are the elemental mechanical interpolation matrix, the mechanical discrete elemental gradient operator, the elemental thermal interpolation matrix, and the thermal discrete elemental gradient operator for scalars.

In a similar manner, the global mass and capacitance matrices are also usually obtained by assemblage of their elemental counterparts as
\begin{align}
  \mathbf M &\equiv \assemble_{e=1}^{n_\text{elem}} \mathbf M^{(e)},\\
  \mathbf C &\equiv \assemble_{e=1}^{n_\text{elem}} \mathbf C^{(e)},\\
  \mathbf K &\equiv \assemble_{e=1}^{n_\text{elem}} \mathbf K^{(e)},
\end{align}
where the elemental mass matrices in the material description are defined as
\begin{highlight}
  \begin{equation}
    \mathbf M^{(e)} = \int_{^h\Omega^{(e)}} \rho_0 {\mathbf N^u}^T\mathbf N^u\ud v,
  \end{equation}
\end{highlight}
the elemental thermal capacitance matrices as
\begin{highlight}
  \begin{equation}
    \mathbf C^{(e)} = \int_{^h\Omega^{(e)}} \rho_0 C_\mathrm{V} {\mathbf N^\theta}^T\mathbf N^\theta\ud v,
  \end{equation}
\end{highlight}
and the elemental thermal stiffness matrix as
\begin{highlight}
\begin{equation}
\mathbf K^{(e)} = \int_{^h\Omega^{(e)}} {\mathbf H^{\theta}}^T\mathbf H^{\theta}\ud v.
\end{equation}
\end{highlight}

% \section{Linearisation}
%
% The equilibrium equation, Equation \eqref{eq:equilibrium_spatial} in a spatial description and Equation \eqref{eq:equilibrium_material} in a material description, is generally nonlinear due to geometrical and/or material nonlinearities.
% The Newton-Raphson Method is an efficient and robust iterative scheme with a quadratic convergence rate often used to solve the equilibrium equation at each time increment, $t_n$.
% The residual of the fully discretized balance of linear momentum is defined for an iteration step \(i\) of the Newton-Raphson method as
% \begin{equation}
% \mathbf{r}(\mathbf{u}_{n+1}^{i})=\mathbf{M} \ddot{\mathbf u}_{n+1}^{i}+\mathbf f^\text{\;int} (\mathbf{u}_{n+1}^{i})-\mathbf{f}_{n+1}^\text{\;ext}.
% \end{equation}
% A Taylor expansion about the current solution \(\mathbf{u}_{n+1}^{i}\) is performed, discarding all terms of  higher order than one, yielding the linearised form
% \begin{equation}
% \operatorname{Lin} \mathbf{r}(\mathbf{u}_{n+1}^{i})=\mathbf{r}(\mathbf{u}_{n+1}^{i})+\underbrace{\left.\frac{\partial \mathbf{r}(\mathbf{u}_{n+1})}{\partial \mathbf{u}_{n+1} }\right|^{i} }_{\mathbf{K}(\mathbf{u}_{n+1}^{i})} \delta \mathbf{u}.
% \end{equation}
% with the dynamic effective tangential stiffness matrix \(\mathbf{K}(\mathbf{u}_{n+1}^{i})\).
% The linearisation of the internal forces included in \(\mathbf{K}\) is known as the tangential stiffness matrix \(\mathbf{K}_T\), which is defined as
% \begin{equation}
% \mathbf{K}_{T}^{i}=\left.\frac{\partial \mathbf{f}^{\text{\;int}} }{\partial \mathbf{u}_{n+1}}\right|^{i}.
% \end{equation}
% Equilibrium is achieved if
% \begin{equation}
% \operatorname{Lin} \mathbf{r}(\mathbf{u}_{n+1}^{i}) = \mathbf{0},
% \end{equation}
% so that a linear system of equation is given by
% \begin{equation}
% \mathbf{K}(\mathbf{u}_{n+1}^{i}) \delta \mathbf{u}=-\mathbf{r}\left(\mathbf{u}_{n+1}^{i}\right).
% \end{equation}
% Thus, a new solution of the displacement increment \(\delta \mathbf{u}\) for current iteration step \(i+1\) is determined, and the final displacement solution of time step \(n+1\) is obtained via updating
% \begin{equation}
% \mathbf{u}_{n+1}^{i+1}=\mathbf{u}_{n+1}^{i}+\delta \mathbf{u}.
% \end{equation}
% A solution of \(t_{n+1}\) is found, i.e. an equilibrium state is reached and \(\mathbf{u}_{n+1}=\mathbf{u}_{n+1}^{i+1}\), if prescribed, user-defined convergence criteria are fulfilled.
%
