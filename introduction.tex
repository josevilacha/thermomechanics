\chapter{Introduction}

This report is concerned with computational techniques for the numerical simulation of thermomechanical problems.
The purpose of this study and its goals are explained in the current introductory chapter.
A summary of the document is also provided.

\section{Motivation}


Thermomechanics is widely acknowledged as a crucial physical phenomenon in engineering applications and a highly sought effect in computational models.
Thermomechanical interaction is key to adequately describe a large variety of technological processes, including sintering and material removal procedures.
To include it in a numerical simulation toolbox, the problem has to be correctly formulated in a thermodynamically consistent way.
Having achieved an appropriate formulation of the problem, the Finite Element Method can be used to solve it.

After the spatial discretization obtained by the FEM has been provided, the complete thermomechanical problem can still be solved by employing two strategies: a monolithic approach or a partitioned approach.
In the former, the discretized equations representing the momentum balance and energy conservation are solved simultaneously.
In the latter, the problems are solved independently, solving the mechanical problem at a fixed temperature and then the thermal problem at a fixed configuration, taking the so-called isothermal split as an example.
With this last scheme, one can still find the loosely or explicit approaches and the strongly coupled or implicit approaches as alternatives.
Each of these strategies has its benefits and drawbacks, with the particular choice of one over the others depending on the particulars of the development environment.

\section{Computational Framework}

All the numerical simulations based on the Finite Element Method (FEM) are held in the in-house Fortran (IBM Mathematical Formula Translation System) program LINKS (Large Strain Implicit Non-linear Analysis of Solids Linking Scales), a multi-scale finite element code for implicit infinitesimal and finite strain analyses of hyperelastic and elastoplastic solids, that is continuously developed by the CM2S research group (Computational Multi-Scale Modeling of Solids and Structures) at the Faculty of Engineering of University of Porto.

In the present work, the author contributes to the addition of a suitable coupling environment for the partitioned solution of coupled fields and a thermal solver based on the Finite Elements Method.
Appropriate nonlinear solvers are also added as implicit solution strategies for the coupled thermomechanical problem.

\section{Objectives}

The main goals of this work are:
\begin{itemize}
    \item To describe in a thermodynamically consistent way the thermomechanical problem;
    \item To develop and validate a thermal solver based on the Finite Element Method;
    \item To provide a thorough overview of the available methods for the solution of coupled problems, in particular, the thermomechanical problem;
    \item To validate the thermomechanical solver and compare the strongly coupled partitioned strategies available in the literature.
\end{itemize}

\section{Document structure}

The remainder of this document is structured as follows:

\paragraph{Chapter \ref{ch:continuum_mechanics} - Continuum Thermomechanics}\mbox{}\\
This chapter covers the notions required to explain how a solid responds to thermal and mechanical loads under large deformations, including the conservation laws that guarantee mechanical equilibrium and energy conservation.
Additionally, the application of thermodynamics with internal variables is discussed, along with the resulting inferences about the constitutive behavior of the material that makes up the solid.

\paragraph{Chapter \ref{ch:mechanical_problem} - Mechanical problem}\mbox{}\\
This chapter presents the strictly mechanical problem, including the constitutive initial value problem, the weak form of the momentum balance equations, and the corresponding mechanical initial boundary value problem.
A brief description of the application of the Finite Element Method to this problem is also included.

\paragraph{Chapter \ref{ch:thermal_problem} - Thermal problem}\mbox{}\\
This chapter presents the strictly thermal problem, including the constitutive law for the heat flux, the weak form of the energy conservation equation, and the corresponding thermal initial boundary value problem.
A brief description of the application of the Finite Element Method to this problem is also included.

\paragraph{Chapter \ref{ch:thermo_mechanical_problem} - Thermomechanical problem}\mbox{}\\
This chapter presents the thermomechanical problem, including the constitutive initial value thermomechanical problem, the weak form of the energy conservation equation and the momentum balance equations, and the corresponding thermomechanical initial boundary value problem.
A brief description of the application of the Finite Element Method to this problem is also included.

\paragraph{Chapter \ref{ch:val_therm_solver} - Validation results for the thermal solver}\mbox{}\\
This chapter details the validation results for the thermal solver using as references the \cite{DINEN1991_1_2} and \cite{NAFEMSbenchmarks}.
It includes both transient effects and boundary conditions such as natural convection and radiation.

\paragraph{Chapter \ref{ch:sol_proc_coupl_fields} - Solution procedures for coupled fields}\mbox{}\\
This chapter presents an overview of the solution procedures for coupled problems.
It includes monolithic schemes and partitioned schemes, both explicit and implicit approaches.
An evaluation and discussion of the different methods are provided.

\paragraph{Chapter \ref{chapter:implicit_meth} - Implicit solution methods for coupled fields}\mbox{}\\
This chapter provides a thorough description of the available implicit methods.
It rests on recasting the problem as a simple root-finding problem for a set of nonlinear equations.
The methods presented are the fixed-point method, the underrelaxation method, the Aitken relaxation, the Broyden-like family of methods, the Newton-Krylov methods, and the polynomial vector extrapolation methods in cycling mode.
The number of residual evaluations, the memory requirements, the computational complexity, and the ease of implementation are all discussed for each approach.

\paragraph{Chapter \ref{ch:val_acc_techniques} - Numerical results for the implicit coupling schemes}\mbox{} \\
This chapter covers the validation results for the thermomechanical solver and the implicit schemes explored in this work.
Each class of implicit methods is examined for efficiency, including as a function of coupling strength, and the best methods in each category are contrasted.
The use of polynomial predictors is also explored.

\paragraph{Chapter \ref{chapter:conclusions}}\mbox{}\\
This chapter presents the conclusions reached in this work, and some future research directions are suggested.

\newpage\null\thispagestyle{blank}\newpage
